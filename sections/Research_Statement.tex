My research focuses on the development and characterization of multiscale and multiphysics science and engineering software tools and abstractions that can be applied on leadership class supercomputers and novel hardware architectures. The tooling and numerical methods my group develops have broad applicability and, through extensive collaborations, are applied to support the development of magnetic confinement fusion reactors, thermal simulations of microelectronic devices, and multiscale simulations of fibrous materials.

My research is at the cutting-edge of computing, often making use of leadership class computing resources. These supercomputers are akin to a telescope, or other large scientific instrument in terms of capital investment, and proposal-driven allocations.

The following is a brief description of my groups capabilities and future looking directions. 

My PhD thesis focused on understanding the physics of fibrous materials with a particular focus on enabling multiscale simulations through understanding size-effects and the development of a GPU accelerated multiscale infrastructure \cite{mersonNewOpensourceFramework2024}.

Through two years of funding from IBM \cite{IBMAIRC2023,IBMFCRC2024}, this infrastructure has been extended to support thermal simulations of microelectronic devices \cite{ithermModeling2025,mersonItherm2026}

Multiscale simulation methods:
- microelectronics
- fibrous materials
- linkages to AI/ML methods?

Code Coupling methods:
- Fusion:
Support for complex geometries
- linkages to AI/ML

My research program provides key capabilities that accelerate scientific discovery and support critical simulations in growing areas of national need.